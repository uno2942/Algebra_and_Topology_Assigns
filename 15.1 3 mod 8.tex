\documentclass[12pt]{article}
\usepackage{graphicx, amssymb}
\usepackage{amsmath}
\usepackage{amsfonts}
\usepackage{amsthm}
\usepackage{kotex}
\usepackage{bm}
\usepackage{xcolor}
\usepackage{mathrsfs}
\usepackage{tikz-cd}
\usepackage{physics}
\usepackage{enumitem}
\usepackage{mathtools}


\textwidth 6.5 truein 
\oddsidemargin 0 truein 
\evensidemargin -0.50 truein 
\topmargin -.5 truein 
\textheight 8.5in
\setlist[enumerate]{align=left}

\DeclareMathOperator{\cc}{\mathbb{C}}
\DeclareMathOperator{\zz}{\mathbb{Z}}
\DeclareMathOperator{\rr}{\mathbb{R}}
\DeclareMathOperator{\bA}{\mathbb{A}}
\DeclareMathOperator{\fra}{\mathfrak{a}}
\DeclareMathOperator{\frb}{\mathfrak{b}}
\DeclareMathOperator{\frm}{\mathfrak{m}}
\DeclareMathOperator{\frp}{\mathfrak{p}}
\DeclareMathOperator{\slin}{\mathfrak{sl}}
\DeclareMathOperator{\Lie}{\mathsf{Lie}}
\DeclareMathOperator{\Alg}{\mathsf{Alg}}
\DeclareMathOperator{\Spec}{\mathrm{Spec}}
\DeclareMathOperator{\End}{\mathrm{End}}
\DeclareMathOperator{\rad}{\mathrm{rad}}
\newcommand{\id}{\mathrm{id}}
\newcommand{\Hom}{\mathrm{Hom}}
\newcommand{\Sch}{\mathbf{Sch}}
\newcommand{\I}{\mathcal{I}}
\newcommand{\Z}{\mathcal{Z}}
\newcommand{\tensorp}{\bigotimes}
\newcommand{\isomorp}{\cong}
\newcommand{\Ring}{\mathbf{Ring}}
\newtheorem{lemma}{Lemma}
\newtheorem{theorem}{Theorem}
\newtheorem{proposition}[theorem]{Proposition}


\begin{document}


%%\title{Financial Management HW - 1} 
%%\author{SungBin Park, 20150462}
%%\maketitle
\section*{Dummit\&Foote Abstract Algebra section 15.1 Solutions for Selected Problems by 3 mod(8)}
Author: SungBin Park, Physics, 20150462
\begin{enumerate}
\item[3.] I'll a lemma to use.
\begin{lemma}
For a field $k$, $k[x]$ has infinitely many primes, which is irreducible.
\end{lemma}
\begin{proof}
Since $k$ is a field, $k{x}$ is a P.I.D, so any prime ideals are maximal ideal, and a prime element is a irreducible element. Let there are finitely many prime element: $p_1(x),p_2(x)\ldots, p_n(x)$ for $n\in \mathbb{N}$, and let $p(x)=\prod_{i=1}^n p_i(x)+1$. Since $p(x)$ is not a prime element, the $(p(x))$ should be contained in one of the maximal ideal $(p_i(x))$ for some $i$. WLOG, let $i=1$. However, $(p(x),p_1(x))=1$, so it is contradiction to $p(x)\in (p_i(x))$. Therefore, there are infinitely many primes in $k[x]$. Since primes are irreducibles in P.I.D, the last statement follows.
\end{proof}
\begin{lemma}
...
\end{lemma}
\begin{proof}
Since $k$ is a field, $k{x}$ is a P.I.D, so any prime ideals are maximal ideal, and a prime element is a irreducible element. Let there are finitely many prime element: $p_1(x),p_2(x)\ldots, p_n(x)$ for $n\in \mathbb{N}$, and let $p(x)=\prod_{i=1}^n p_i(x)+1$. Since $p(x)$ is not a prime element, the $(p(x))$ should be contained in one of the maximal ideal $(p_i(x))$ for some $i$. WLOG, let $i=1$. However, $(p(x),p_1(x))=1$, so it is contradiction to $p(x)\in (p_i(x))$. Therefore, there are infinitely many primes in $k[x]$. Since primes are irreducibles in P.I.D, the last statement follows.
\end{proof}
\begin{proof}
Let $k(x)$ is finitely generated k-algebra, so $k(x)=\left(\frac{a_1(x)}{b_1(x)},\frac{a_2(x)}{b_2(x)},\ldots, \frac{a_n(x)}{b_n(x)}\right)$ for some $n\in \mathbb{N}$, $a_i(x),b_i(x)\in k[x]$, $a(x)$ and $b(x)$ relatively prime. Since $k[x]$ is U.F.D, let $p_{ij}(x)$ be irreducible factors of $a_i(x)$, i.e., $a_i(x)=u_i\prod_{j=1}^{m_i}p_{ij}(x)$ for some unit $u_i$. By the lemma above, there is a irreducible $p(x)$ such that $p(x)\notin (p_{11}(x), \ldots, p_{1m_i}(x),p_{21}(x),\ldots, p_{nm_n}(x))$.

Let $\frac{1}{p(x)}\in k(x)=\left(\frac{a_1(x)}{b_1(x)},\frac{a_2(x)}{b_2(x)},\ldots, \frac{a_n(x)}{b_n(x)}\right)$, then $\frac{1}{p(x)}\in k\left[\frac{a_1(x)}{b_1(x)},\frac{a_2(x)}{b_2(x)},\ldots, \frac{a_n(x)}{b_n(x)}\right]$. Since $\frac{1}{p(x)}$ is not constant, and right hand side is polynomial, it would be written as $\frac{r(x)}{t(x)}$ for some $r(x),t(x)\in k[x]$. Note that $t(x)$ is at least degree 1 polynomial having a irreducible factor in $p_{ij}(x)$ for some $i,j$.... To meet $\frac{1}{p(x)}=\frac{r(x)}{t(x)}$, $t(x)=p(x)r(x)$ but it means  $r(x)$ is divided by $p(x)$ which is contradiction to assumption. Therefore, $k(x)$ is not finitely generated k-algebra
\end{proof}
\item[11.]\begin{enumerate}
\item[(a)] Let $S$ be the set of not finitely generated ideals of $R$. 
\item[(b)] Since $I$ is not a prime ideal, there exists $a,b\notin I$ such that $ab\in I$. Let $J_1=(I,a)$, $J_2=(I,b)$. Then, $I\subset J_1,J_2$ and $J_1J_2=(I^2, aI, Ib, ab)\subset I$ since $I^2\subset I$, $ab\in I$. Since $J_1,J_2\supset I$, $J_1$ and $J_2$ are finitely generated by the maximality of $I$. Let $J_1=(a_1, \ldots, a_n)$, $J_2=(b_1, \ldots, b_m)$, then $J_1J_2=(a_1b_1, a_1,b_2, \ldots, a_1b_m,a_2b_1, \ldots, a_nb_m)$. Therefore, $J_1J_2$ is finitely generated.
\item[(c)] $R/I$ is a Noetherian ring, I can show that $I/J_1J_2$ is finitely generated by proving $J_1/J_1J_2$ is finitely generated $R/I$ module according to problem 8. Since $J_1=(a_1, \ldots, a_n)$, we can represent $J_1/J_1J_2=(\overline{a_1}, \ldots, \overline{a_n})$ where $\overline{a_i}=a_i+J_1J_2$. Let's represent $m\in J_1/J_1J_2$ by $m+J_1J_2=\sum\limits_{i=1}^n (r_i+I)(a_i+J_1J_2)$ for some $r_i\in R$. If we multiply $(r+I)$, $r\in R$, to this module, 
\begin{equation*}
\begin{split}
(r+I)\left(\sum\limits_{i=1}^n \overline{r_ia_i}\right)&=\sum\limits_{i=1}^n (r+I)(r_ia_i+J_1J_2) \\
&=\sum\limits_{i=1}^n r r_ia_i+rJ_1J_2+Ir_ia+IJ_1J_2 \\
&=\sum\limits_{i=1}^n r r_ia_i+J_1J_2 \in J_1/J_2J_2.
\end{split}
\end{equation*}
...
Therefore, $J_1/J_1J_2$ is finitely generated $R/I$ module and hence Noetherian. It implies $I/J_1J_2$, which is submodule of $J_1/J_1J_2$, is finitely generated.
\item[(d)] Since $I/J_1J_2$ is finitely generated $R/I$ module, let $I/J_1J_2=(\overline{i_1}, \ldots, \overline{i_l})$ where $i_i\in I$, $\overline{i_i}=i_1+J_1J_2$. I'll claim that $I=(i_1, \ldots, i_l, a_1b_1, \ldots, a_nb_m)$ as an ideal of $R$. It is trivial that  $I\supset(i_1, \ldots, i_l, a_1b_1, \ldots, a_nb_m)$ so I'll show the converse. Let $i\in I$, then $i+J_1J_2=\sum\limits_{j=1}^l (r_j+I)(i_j+J_1J_2)$ for some $r_j\in R$. Then, $\sum\limits_{j=1}^l (r_j+I)(i_j+J_1J_2)=\sum\limits_{j=1}^l r_ji_j+r_jJ_1J_2+Ii_j+IJ_1J_2=\sum\limits_{j=1}^l r_ji_j+J_1J_2$ since $Ii_j\in \subset I^2\subset J_1J_2$. Therefore, $i=\sum\limits_{j=1}^l r_j i_j-\sum\limits_{n,m} r_{ks}a_kb_s$. Hence $I=(i_1, \ldots, i_l, a_1b_1, \ldots, a_nb_m)$ and $I$ is finitely generated, which is contradiction to (a).
\end{enumerate}
\item[19.]
\item[27.]
\item[35.]
\item[43.]
\end{enumerate}

\end{document}